% The name of this file: cddman.tex
% written by by Komei Fukuda
% December 14, 1993
% Revised June 21, 1996
%
\documentclass[11pt]{article}
\usepackage{html}
\renewcommand{\baselinestretch}{1}
\renewcommand{\arraystretch}{1.2}
\setlength{\oddsidemargin}{8mm}
\setlength{\textwidth}{16cm}
\setlength{\topmargin}{-5mm}
\setlength{\textheight}{235mm}
%\setlength{\headsep}{0in}
%\setlength{\headheight}{0pt}
\pagestyle{plain}

\begin{document}
\title{cdd+ Reference Manual}
\author{
Komei Fukuda \\
Institute for Operations Research\\
ETH-Zentrum, CH-8092 Zurich, Switzerland\\
fukuda@ifor.math.ethz.ch \\and\\
GSSM, University of Tsukuba, Tokyo, Japan\\
fukuda@gssm.otsuka.tsukuba.ac.jp
}
\date{ (cdd+ Version 0.74,  June 17, 1995)}

\maketitle

\section{What's new?}
If you are a new user of cdd+ please read Section \ref{INTRODUCTION} (Introduction) first.
The version 0.74 has several improvements.  
\begin{description}
\item[(1)] The first one, which is not an additional
function, is some changes in default output file names.  In particular, each different
function (transformation) generates a file whose name extension
 represents the transformation, see Section~\ref{HOWTO}.
The changes are made to reduce possible confusion when one uses cdd+
for many different purposes.  We are about to release
a collection of volume computation codes \cite{befl-cevcm-95} for convex polytopes 
for which our new file naming rules are important. 
\item[(2)] Linear programming algorithms have been updated, and now the default
algorithm (dual simplex method) works efficiently for dense LP's with many constraints
and few variables.
\item[(3)] Because of improvement (2), the redundancy removal functions (facet\_listing,
vertex\_listing) and the tope\_listing function work much better than before.
In case that input contains many redundant inequalities or points,
the vertex enumeration or convex hull computation might be more
efficient if one removes redundancy from input first.
\item[(4)] The floating-point executable cddf+ can now read rational
data (of form $p / q$) and compute with floating-point arithmetic.
\item[(5)] We added new options ``facet\_listing\_external'' and ``vertex\_listing\_external'' to
check with a large external file whether adding each of the external input data
 to the input file data changes the input polytope or not.  These are useful
when one wish to remove redundancy from a large input data
(say, of few million inequalities or points in moderate dimensions) that
cannot fit in physical memory of usual computers.
\end{description}

\section{Introduction} \label{INTRODUCTION}

The program  cdd+  (cdd, respectively)   is 
a C++  (ANSI C) implementation of 
the Double Description Method~\cite{mrtt-ddm-53}
for generating all vertices (i.e. extreme points)
and extreme rays of a general 
convex polyhedron given by a system of linear inequalities:
\[
   P = \{ x  \in R^d:  A  x  \le  b \}
\]
where $A$ is an $m \times d$ real matrix and $b$ is a real
$m$ dimensional vector.   See, \cite{fp-ddmr-95} for
an efficient implementation of the double description
method which is employed in cdd+.


One useful feature of  cdd+ (and cdd) is its capability
of handling the dual (reverse)  problem without any transformation
of data.  The dual problem is known to be the 
{\em (convex) hull problem\/} which
is to obtain a linear inequality representation
of a convex polyhedron given as the Minkowski sum of 
the convex hull of a finite set of points and the nonnegative
hull of a finite set of points in $R^d$: 
$P = conv(v_1,\ldots,v_n) +  nonneg(r_1,\ldots,r_s)$.
As we see in this manual, the computation can be done
in straightforward manner.  There is one assumption for the input
for hull computation: the polyhedron must be full-dimensional.
See the ``hull'' option in Section~\ref{OPTIONS}.

Besides these basic functions, cdd+ can solve the general
linear programming (LP) problem to maximize (or minimize) a linear
function over polyhedron $P$.   It is useful mainly for solving 
dense LP's with large $m$ (say, up to few hundred thousands) and small $d$ 
(say, up to 30).

The program cdd+ is a C++ program converted from the ANSI C
program cdd.  One major advantage of this C++-version over the C version is
that it can be compiled for both rational (exact) arithmetic and 
floating point arithmetic.  Note that cdd runs on floating
arithmetic only.  Since cdd+ uses GNU g++ library, in particular 
Rational library, one needs a recent (2.6.0 or higher) gcc compiler
and g++-lib.  One should be also warned that the computation
can be considerably  (10 - 100 times or even more) slower if the rational
arithmetic is used.

The program cdd+ (and cdd) reads input and writes output in 
{\em Polyhedra format\/} which was defined by David Avis and
the author.  The program called rs developed by David Avis is
a C-implementation of the reverse search algorithm~\cite{af-pachv-92} 
for the same enumeration purpose, and it conforms to Polyhedra format as well.
Hopefully, this compatibility of the two programs
enables users to use both programs for the same input files
and to choose whichever is useful for their purposes.
From our experiences with relatively large problems,
the two methods are both useful and perhaps complementary
to each other.  In general, the program cdd+ tends to be
efficient for highly degenerate inputs and the program rs
tends to be efficient for nondegenerate or slightly
degenerate problems.

Among the hardest problems that could be
solved (in floating-point arithmetic) by cdd+ is a 
21-dimensional hull problem given by 64 
vertices. This polytope, known as the {\em complete
cut polytope on 7 points\/}, has exactly 116,764 facets
and some of facets contain many vertices. 
It took 205 hours (eight and half days!) for cdd
to compute the facets exactly on a SUN SparkServer 1000.
The input file (ccp7.ine) of this polytope is
included in the distribution.  A considerably easier
problem is ccc7.ine which is a variation of the problem 
(see e.g. \cite{g-afccn-90}).

The size of an input file hardly indicates the degree of 
hardness of its vertex/ray enumeration.  While this program
can handle a highly degenerate problem (prodmT5.ine) with 
711 inequalities in  19 dimension quite easily with
the computation time 1-2 minutes on a fast workstation, 
a 8-dimensional problem (mit729-9.ine) with 729 inequalities
can be extremely hard.  It takes two days to compute all
(only 4862) vertices by a fast SUN SparkServer 1000.  The latter problem arises
from the ground state analysis of a ternary alloy model, see \cite{cgaf-gstfl-94}.
Both input files are included in the distribution.  

Although the program can be used for nondegenerate inputs,
it might not be very efficient.  For nondegenerate inputs, 
other available programs, such as the reverse search code lrs or
qhull (developed by the Geometry Center),
might be more efficient.  See Section~\ref{CODES} 
for pointers to these codes.  
The paper \cite{abs-hgach-95} contains many interesting results on polyhedral
computation and experimental results on cdd+, lrs, qhull and porta.

This program can be distributed freely under the GNU GENERAL PUBLIC LICENSE.
Please read the file COPYING carefully before using.

I will not take any responsibility of any problems you might have
with this program.  But I will be glad to receive bug reports or suggestions
at the e-mail addresses above.  Finally, if cdd+ turns out to be useful, 
please kindly inform to me  what purposes cdd has been used for. 
I will be happy to include a list of applications in future
distribution  if I receive  enough replies.
The most powerful support for free software development
is user's appreciation and collaboration.

\section{Polyhedra H- and V-Formats} \label{FORMAT}
\bigskip
Every convex polyhedron has two representations, one as
the intersection of finite halfspaces and the other
as Minkowski sum of the convex hull of finite points
and the nonnegative hull of finite directions.  These are
called H-representation and V-representation, respectively.

Naturally there are two basic Polyhedra formats, 
H-format for  H-representation and V-format for
V-representation.    These two formats are designed
to be almost indistinguishable, and in fact, one can
almost pretend one for the other.   There is some asymmetry
arising from the asymmetry of two representations.

First we start with the halfspace representation.
Let $A$ be an $m \times d$ matrix, and let $b$ be a column $m$-vector.
The Polyhedra format  ({\em  H-format} )  of 
the system $ A x \le b$ of $m$ inequalities in $d$ variables
$x =(x_1, x_2, \ldots, x_d)^T$ is

\begin{tabular}{ccl}
\\ \hline
\multicolumn{3}{l} {various comments}\\
\multicolumn{3}{l} {\bf begin}\\
 $m$ & $d+1$ & {\bf numbertype}\\
 $b$ & $-A$ \\
\multicolumn{3}{l} {\bf end}\\
\multicolumn{3}{l} {various options} \\ \hline
\end{tabular}

\bigskip
\noindent
where numbertype can be one of integer, rational or real.
When rational type is selected, each component
of $b$ and $A$ can be specified by the usual integer expression 
or by the rational expression ``$p / q$''  or  ``$-p / q$'' where
$p$ and $q$ are arbitrary long positive integers (see the example
input file rational.ine).
There is one restriction in the current polyhedra format: 
the last $d$ rows must determine
a vertex of $P$.  The program cdd+ does not care whether
this condition is satisfied, as long as the polyhedron
contains at least one vertex.  But one can specify that 
the last $(d+1)$ rows be chosen as the initial set of
$(d+1)$ rows for the double description algorithm.
See {\bf initbasis\_at\_bottom} option in Section~\ref{OPTIONS}.

Now we introduce  Polyhedra  {\em V-format}.  Let $P$ be 
represented by $n$ extreme points and $s$ rays as 
$P = conv(v_1,\ldots,v_n) +  nonneg(r_1,\ldots,r_s)$.
Then the Polyhedra V-format for $P$ is defined as

\begin{tabular}{ccl}
\\ \hline
\multicolumn{3}{l} {various comments produced by a program}\\
\multicolumn{3}{l} {\bf begin}\\
 $n+s$ & $d+1$ & {\bf numbertype}\\
 $1$ & $v_1$  & \\
 $\vdots$ & $\vdots$  & \\
 $1$ & $v_n$  & \\
 $0$ & $r_1$  & \\
 $\vdots$ & $\vdots$  & \\
 $0$ & $r_s$  & \\
\multicolumn{3}{l} {\bf end}\\
\multicolumn{3}{l} {hull}\\  \hline
\end{tabular}

\bigskip
\noindent
Here we do not require that
vertices and rays are listed
separately; they can appear mixed in arbitrary
order.   The option ``hull'' must be set for every V-format.


It is strongly suggested to use the following rule for naming
H-format files and V-format files:   
\begin{description}
\item[(a)] use the filename  extension ``.ine'' for H-files (where ine stands for inequalities), and 
\item[(b)]  use the filename  extension ``.ext'' for V-files (where ext stands for extreme points/rays). 
\end{description}

The program cdd+ does two transformations, one from an H-format
to a V-format, and the reverse.    While an input file (in H-format or V-format)
can have redundant information, cdd+ output a minimal representation
(in V-format or H-format).

For example, let $P$ be the following unbounded 3-dimensional 
H-polyhedron given by
\[
   P = \{ x  \in R^3:
    1\le x_1 \le 2, \; 1 \le x_2 \le 2, \; 1 \le x_3,  \; x_1 + x_2 \le 4 \},
\]
which is a 3-cube without one ``lid''.   The last inequality is redundant
because it is implied by $x_1 \le 2$ and $x_2 \le 2$.
For finding all
vertices and extreme rays, the input file for cdd+ is

\begin{verbatim}
file name: ucube.ine
3 cube without one "lid"
begin
    6      4    integer
  2   -1   0   0
  2    0  -1   0
 -1    1   0   0
 -1    0   1   0
 -1    0   0   1
  4   -1  -1   0
end
incidence
adjacency
input_adjacency
input_incidence
\end{verbatim}

The meaning of options ``incidence'', ``adjacency''
``input\_adjacency'' and ``input\_incidence'' 
will be explained in Section~\ref{OPTIONS}.
After you run cddf+ (the floating-arithmetic version of cdd+) 
with this input file, you will get
an output file ucube.ext  which is the minimal V-representation
of the polyhedron:

\begin{verbatim}
* cdd+: Double Description Method in C++:Version 0.74 (June 17, 1996)
* Copyright (C) 1996, Komei Fukuda, fukuda@ifor.math.ethz.ch
* Compiled for Floating-Point Arithmetic
*Input File:ucube.ine(6x4)
*HyperplaneOrder: LexMin
*Degeneracy preknowledge for computation: None (possible degeneracy)
*Vertex/Ray enumeration is chosen.
*Output adjacency file is requested.
*Input adjacency file is requested.
*Output incidence file is requested
*Input incidence file is requested.
*Computation completed at Iteration 6.
*Computation starts     at Sun Jun 17 16:40:32 1996
*            terminates at Sun Jun 17 16:40:32 1996
*Total processor time = 0 seconds
*                     = 0h 0m 0s
*FINAL RESULT:
*Number of Vertices =4, Rays =1
begin
5  4  real
 1 2 1 1
 1 1 1 1
 1 1 2 1
 1 2 2 1
 0 0 0 1
end
hull
\end{verbatim}

The output shows that the polyhedron has four vertices
$(2,1,1)$, $(1,1,1)$, $(1,2,1)$, $(2,2,1)$ and
only one extreme ray $(0,0,1)$.  The comments contain
information on the name of input file, and the options
chosen to run the program which will be explained in
the next section.  

Now, if you run cdd+ with this output file ucube.ext,
cdd+ will perform the convex hull operation to recover 
essentially the original input inequality system.  Note that
this back-and-forth transformation of a polyhedron works
only when the polyhedron is full dimensional and contains
at least one vertex.

\section{Options}  \label{OPTIONS}

The following options are available for cdd.  These options are
set if they appear in input file after the ``end'' command.
Independent options can be set simultaneously, but each option
must be written separately in one line, and  two options
should not be written in one line.  When two or more non-independent
options are specified, the last one overrides the others.
Also note that options are case-sensitive.

\begin{description}

\item[hull] option \\
When this option is chosen, the program cdd
will do the reverse operation.   That is, the input is assumed to
be a set of points and directions (rays).  When this option is set,
it is required that each data line must start with 
either ``1'' or ``0'', meaning points and rays, respectively.
More specifically, if the input file is of form

\begin{tabular}{ccl}
\\ \hline
\multicolumn{3}{l} {comments}\\
\multicolumn{3}{l} {\bf begin}\\
 $n+s$ & $d+1$ & {\bf numbertype}\\
 $1$ & $v_1$  & \\
 $\vdots$ & $\vdots$  & \\
 $1$ & $v_n$  & \\
 $0$ & $r_1$  & \\
 $\vdots$ & $\vdots$  & \\
 $0$ & $r_s$  & \\
\multicolumn{3}{l} {\bf end}\\
\multicolumn{3}{l} {\bf hull} \\ \hline
\end{tabular}

\noindent
Then the input is interpreted as the polyhedron in $R^d$:\\
$P = conv(v_1,\ldots,v_n) +  nonneg(r_1,\ldots,r_s)$ 
and the output will be a minimal system of linear inequalities
to represent $P$. 

\item[verify\_input] option\\
When this option is chosen, the program will
output the input problem as cdd+ interpreted.  
The default output  file is ``*.solved''. 
This option helps user to verify what problem
is actually solved.  The default for this option is off.
See the sample files verifyinput1.ine and verifyinput2.ine

\item[dynout\_off] option\\
When this option is chosen, the program will
not output vertices and rays to the CRT in real time.  
The default is dynout\_on.

\item[stdout\_off] option\\
 When this option is chosen, the program will not
output any progress report of computation (iteration number. etc).
The default is stdout\_on.

\item[logfile\_on] option\\
 When this option is chosen, the program will output
to a specified file (*.ddl) some information on the computation history.
This can be useful when the user does not know which hyperplane order
(mincutoff, maxcutoff, mixcutoff, lexmin, lexmax, minindex, random)
is efficient for computation.

\item[incidence, input\_incidence,  \#incidence] options\\
When the {\bf incidence} option is selected, the incidence relation for
each {\bf output} with respect to input will be generated.    The default filename
is *.icd  if {\bf output} is inequalities (i.e. *.ine), and *.ecd  
\footnote {In earlier versions, *.icd was used for the new *.ecd file.}
if {\bf output} is extreme points and
rays (i.e. *.ext).

When the {\bf input\_incidence} option is selected, the incidence relation for
each {\bf input} with respect to output will be generated.    The default filename
is *.icd  if {\bf input} is inequalities (i.e. *.ine), and *.ecd  
if {\bf input} is extreme points and
rays (i.e. *.ext).  This option is newly added to Version 0.74.

Here, an extreme point is said to be 
{\em incident with\/} an inequality if the inequality is satisfied by equality.
An extreme ray $r$ is said to be {\em incident with\/} 
an inequality $a^T \; x \le b$ if  $a^T \; r = 0$.  

For example,
since the incidence option was set for the example input file ucube.ine in
the previous section, the program outputs the following ucube.ecd file:
\begin{verbatim}
*Incidences of output(=vertices/rays) and input (=hyperplanes)
*   for each output, #incidence and the set of hyperplanes containing it
*   or its complement with its cardinality with minus sign
*cdd input file : ucube.ine  (6 x 4)
*cdd output file: ucube.ext
begin
  5  6  7
 3 :  1 4 5
 3 :  3 4 5
 3 :  2 3 5
 -3 :  3 4 7
 -1 :  5
end
\end{verbatim}
After ``begin'', there are three numbers $5 \quad 6 \quad 7$.
The first number $5$ is a number of output (vertices and rays).
The next number $6$ is $m$, the number of inequalities in the input file.
The last number $7$ is usually $m+1$, and $m$ if the input linear inequality
system is homogeneous (i.e., has zero RHS) or the hull option is chosen.
The number $m+1$ corresponds to the infinity constraint which is added
for vertex/ray enumeration when the input system is not homogeneous.

The incidence data starts right after these three numbers.
At each line, the cardinality of incident inequalities and
the list of their indices are given.  There is an exception that, when
there are more incident inequalities than non-incident ones, then the program
outputs the list of non-incident inequalities with its
size with negative sign.  This is to save space of output.

For example, the first output line $3 \; : \; 1 \; 4 \; 5$ 
corresponds to the
first vertex of ucube.ext file in previous section, that is, 
the vertex $(2, 1, 1)$.  The first number $3$ is simply the number
of incident inequalities and the rest is the indices of
those inequalities, and so the 1st, 4th and 5th inequalities are
satisfied by equality at this vertex.   The last output
$-1 \; : \; 5$ corresponds to the ray  $(0,0,1)$.  Since all inequalities
except the last (5th) inequality are incident with this ray,
the output is the  (shorter) complementary list with its cardinality (=1) with negative
sign.  Note that the full list would be $6 \; : \; 1 \; 2 \; 3 \; 4 \; 6 \; 7$, where
$6$ is the infinity plane.  One can ignore the infinity
plane for some purposes,  but for analyzing the combinatorial 
structure of polyhedra, it is
very important information.

Also, since the input\_incidence option was set for the example input file ucube.ine in
the previous section, the program outputs the following ucube.icd file:
\begin{verbatim}
*cdd input file : ucube.ine (6 x 4)
*cdd output file: ucube.ext
*Incidence of input (=inequalities/facets) w.r.t. output (=vertices/rays).
*row 7 is redundant;dominated by: 1 2 3 4 6
*row 6 is redundant;dominated by: 1 2
begin
  6  5  5
 -2 : 2 3
 -2 : 1 2
 -2 : 1 4
 -2 : 3 4
 -1 : 5
 2 : 4 5
 1 : 5
end
\end{verbatim}
After ``begin'', there are three numbers $6 \quad 5 \quad 5$.
The first number $6$ is a number of input (inequalities).
The next number $5$ is $m$, the number of vertices and rays in the output file.
The last number $5$ is always  $m$ for all *.icd files.  The remaining
lines can be interpreted similarly with ucube.ecd file.

The {\bf \#incidence} option can be used when you do not wish
to output the incidence file but to output only the cardinality of incidence
for each output, at the end of each output line.

The incidence files (adjacency file, input\_adjacency as well) 
can be created independently
after *.ext file is created, see ``postanalysis'' option. 
 
\item[nondegenerate] option\\
When this option is set, the program assumes that the input system
is not degenerate, i.e., there is no point in the space $R^d$ satisfying
more than $d$ inequalities of input with equality.
It will run faster with this option, but of course, 
if this option is set for degenerate inputs, it is 
quite possible that the output is incorrect.  
The default is this option being off.

\item[adjacency] option\\
This option can be used when you want to output the adjacency of output.
When the output is the list of vertices and rays, the program will
output the adjacency list.  For the example input ``ucube.ine'',
the following extra file  ``ucube.ead''  \footnote{In earlier versions,
this was ``ucube.adj''}  will be created:
\begin{verbatim}
*Adjacency List of output (=vertices/rays)
*cdd input file : ucube.ine (6 x 4)
*cdd output file: ucube.ext
begin
  5
 1 3 : 2 4 5
 2 3 : 1 3 5
 3 3 : 2 4 5
 4 3 : 1 3 5
 5 4 : 1 2 3 4
end
\end{verbatim}
The first number $5$ is simply the number of outputs of cdd, the
number of vertices and rays in this case.
The second line $ 1 \quad  3 \quad : 2 \quad  4 \quad  5$ says 
that the first output of
ucube.ext file has degree (valency) $3$, and its three neighbors are
2nd, 4th and 5th output.

When the computation is to obtain the hull (inequality system),
the adjacency is of course that of inequalities (i.e. facets).

The adjacency file (incidence file, input\_adjacency file) 
can be created independently
after *.ext file is created, see ``postanalysis'' option.

\item[input\_adjacency] option\\
This option is for outputing the adjacency of input inequalities.
Here, two inequalities are defined to be {\em adjacent\/} if 
they are nonredundant and there is no third input inequality
which is satisfied with equality at all points of the polyhedron
that satisfy the two inequalities with equality. 
In more intuitive language, two inequalities are adjacent
if each determine a facet of the polyhedron and the intersection
of the two facets is not contained in any other facet.

The default file name for this output is *.iad.  This file
lists the redundancy information of input also.  For the example
``ucube.ine'' above, the following ``ucube.iad'' will
be generated:
\begin{verbatim}
*Adjacency List of input (=inequalities/facets)
*cdd input file : ucube.ine (6 x 4)
*cdd output file: ucube.ext
*row 7 is redundant;dominated by: 1 2 3 4 6
*row 6 is redundant;dominated by: 1 2
begin
  7
 1 3 : 2 4 5
 2 3 : 1 3 5
 3 3 : 2 4 5
 4 3 : 1 3 5
 5 4 : 1 2 3 4
 6 0 :
 7 0 :
end
\end{verbatim}
Observe that the 6th inequality and the 
artificially added 7th inequality (infinity)
are found redundant.  The 7th inequality is redundant
because the first four facets intersects at
a single infinity point (corresponding to a unique extreme ray)
and hence the polyhedron has no infinity facet,
although the polyhedron is not bounded.

The input\_adjacency file  can be created independently
after *.ext file is created, see ``postanalysis'' option.

\item[postanalysis] option\\
It is often more desirable to compute the adjacency, 
input\_adjacency, incidence and input\_incidence relations
independently from the main (and often heavy) computation of enumerating all vertices
and extreme rays.  The ``postanalysis'' option can be used together
with ``adjacency'' and/or ``incidence'' options for this purpose  
to create *.adj and/or *.icd files from both *.ine and *.ext files.
If *.ine file contains this option, cdd+ will open the
corresponding *.ext file and output requested *.adj, *.iad and/or *.icd files.
An error occurs when *.ext file does not exist in the current
directory.

\item[lexmin, lexmax, minindex,mincutoff, maxcutoff, mixcutoff, random] options\\
The double description is an incremental algorithm which
computes the vertices/rays of a polyhedron given by some $k$ of
original inequalities from the precomputed vertices/rays of a
polyhedron given by $k-1$ inequalities.  It is observed that
the efficiency of the algorithm depends strongly on how
one selects the ordering of inequalities, although a little
can be said theoretically.
These options are to select the ordering of inequalities to be
added at each iteration, and it is recommended to do small
experiment to select good ordering for a specific type of problems.
Unfortunately, a good ordering depends on the problem and there does not seem 
to be THE BEST ordering for every computation.  From our experiences,
lexmin, lexmax, mincutoff, maxcuoff work quite well in general.

The default is lexmin ordering which simply order inequalities
with respect to lexico-graphic ordering of rows of $(b, -A)$.  The lexmax
is reverse of lexmin.  The mincutoff (maxcutoff) option selects an inequality which
cuts off the minimum (maximum) number of vertices/rays of the $(k-1)$st polyhedron. 
The mixcutoff option is the mixture of mincutoff and maxcutoff which selects
an inequality which cuts off the $(k-1)$st polytope as unbalanced as possible.
The maxcutoff option might be efficient if the input contains
many redundant inequalities (many interior points for hull computation).
The minindex option selects the hyperplanes from the top of
the input.

The random option selects the inequalities in a random order.  This option
must be followed by a random seed which is positive integer (less than
65536).  For example, {\bf random 123} specifies the random option with
the random seed 123. 

\item[initbasis\_at\_bottom] option\\
When this option is set, the program tries to select
the initial set of rows for the double description
method from the bottom of the input.  This means that
if the last (d+1) rows are independent, 
they will be chosen to initiate the algorithm.

This option is {\em not\/} default. The default
follows the same ordering as the ordering
of inequalities chosen.  This means that if {\bf lexmin\/}
is the ordering of inequalities, then the initial 
independent rows
will be chosen sequentially with lexico-min ordering.
There are exceptions when this rule
is not applicable, i.e. when one of mincutoff or maxcutoff
options is chosen. In such cases, {\bf lexmin\/}
ordering will be chosen.

\item[maximize, minimize] options\\
When maximize option is set with an objective vector 
$c_0\: c_1 \: c_2 \ldots c_d$, the program
simply solves the linear program: $\max c_0 + c_1 x_1 + c_2 x_2 +\cdots + c_d x_d$
over the input polyhedron $P$. The grammar is simply

\begin{tabular}{ccl}
\\ \hline
\multicolumn{3}{l} {various comments}\\
\multicolumn{3}{l} {\bf begin}\\
 $m$ & $d+1$ & {\bf numbertype}\\
 $b$ & $-A$ \\
\multicolumn{3}{l} {\bf end}\\
\multicolumn{3}{l} {maximize} \\ 
\multicolumn{3}{l} { $c_0 \quad c_1 \quad c_2 \quad \cdots \quad c_d$ } \\ \hline
\\
\end{tabular}

The minimize option works exactly same way for minimization of
a linear objective function.
See the sample input file ``lptest.ine''.  The program cdd
will output both primal and dual optimal solutions  if the LP
is solvable.  If the LP is infeasible (dual infeasible), then
it will output an evidence.

For the moment,  one can use either the dual simplex
method (option ``dual-simplex'', default) 
or the criss-cross method by Terlaky-Wang.  
The latter method can be specified
by option ``criss-cross'' and is very sensible to the ordering
of inequalities.  The ordering options such as maxindex, lexmin and 
random will affect the behavior of this solver.  Try to use
a different ordering, if the computation takes too much time.

Also, in order to see the intermediate LP sign tableau 
one can use ``show\_tableau'' option.  Also use ``manual\_pivot''
option to select pivots manually.   Of course, these options are 
intended for very small problems.

The minimize and maximize options should be used only in H-representation
(*.ine) files, and the output filename is ``*.lps''.

\item[find\_interior] option\\
When this option is set, the program solves the linear program: 
$\max x_{d+1}$
subject to $ A x + e x_{d+1} \le b$, where $e$ is the
column vector of all $1$'s.  If the optimum value is zero,
the polyhedron has no interior point.  If the optimum value is 
negative then the polyhedron is empty. If the LP is dual inconsistent,
then the polyhedron admits unbounded inscribed balls.  To find any
interior point in this last case, one must add some inequality(ies)
to bound the polyhedron. 

This option should be used only in H-representation
(*.ine) files, and the output filename is ``*.lps''.

\item[facet\_listing, vertex\_listing] options\\
When the option ``facet\_listing''  is set, the program checks for each i-th row of the input 
whether the associated inequality $A_i x \le b_i$ determines a facet
of the polyhedron.    This option should be used only in H-representation
(*.ine) files, and the output filename is ``*.fis''.

When the option ``vertex\_listing''  is set, the program checks for each i-th row of the input 
whether the associated point  $v_i$ determines a facet
of the polyhedron.    This option should be used only in V-representation
(*.ext) files, and the output filename is ``*.vis''.

After *.vis or *.fis file (say test.vis) is obtained,  
one can get the minimal nonredundant
system by using the included gawk script  get\_essential:
\begin{verbatim}
% get_essential <  test.vis >test\_ess.ine
\end{verbatim}
You must have a gnu gawk command accessible at
the current unix directory.  One must edit the new file
test\_ess.ine slightly according to the instruction
written in the file.

\item[facet\_listing\_external, vertex\_listing\_external] options\\
These options can be used to apply facet\_listing and vertex\_listing
with a (possibly large) external file.   When  the option facet\_listing\_external 
is set  with *.ine file, cdd+ will open an external file *.ine.external (in H-format)
and verify for each inequality of the external file whether it changes the 
original polyhedron  (represented by *.ine) if it is added.  The option
vertex\_listing\_external can be set in *.ext file and works similarly.
Since cdd+ reads each line of the external file one by one, the file
can be very large, say of few hundred thousands lines. 

\item[tope\_listing] option\\
When this option is set, the program generates all full-dimensional
regions (which are sometimes called topes) of the arrangement
of hyperplanes $\{ h_i : i = 1,2, \ldots, m \}$, where 
$ h_i = \{ x : A_i x \le b_i \}$.  Each tope will be
represented by its location vector, i.e. 
a sign vector in $\{+, -\} ^m$ whose $i$-component
indicates the (positive or negative) side of the hyperplane $h_i$
the tope is located.  This procedure assumes that the input
polyhedron is full-dimensional and thus the vector of all $+$'s
determines a tope.   This option should be used only in H-representation
(*.ine) files, and the output filename is ``*.tis''.


\item[partial\_enumeration, equality, strict\_inequality] options\\
With partial\_enumeration option (or equivalently equality option), 
one can enumerate only those
vertices and rays that are lying on the set of hyperplanes
associated with specified inequality numbers. If you want
to compute all vertices/rays lying on hyperplanes
associated with $k$ inequalities $i_1, i_2, \ldots, i_k$
($1 \le i_j \le m$), then
the option should be specified as

\begin{tabular}{ccl}
\\ \hline
\multicolumn{3}{l} {various comments}\\
\multicolumn{3}{l} {\bf begin}\\
 $m$ & $d+1$ & {\bf numbertype}\\
 $b$ & $-A$ \\
\multicolumn{3}{l} {\bf end}\\
\multicolumn{3}{l} {partial\_enumeration} \\ 
\multicolumn{3}{l} { $k \quad i_1 \quad i_2 \quad \cdots \quad i_k$ } \\ \hline
\\
\end{tabular}

The {\bf strict\_inequality\/}  option follows the same 
grammar as partial\_enumeration or equality.  
With this, cdd outputs only those vertices and rays not satisfying
any of the specified inequalities with equality.  See the sample files, 
partialtest1.ine and partialtest2.ine. 

These options make no effect on LP maximization or minimization.

\item[preprojection] option\\
This option is for a preprocessing
of orthogonal projection of the polyhedron to the subspace of
$R^d$ spanned by a subset of variables.  
That is, if the inequality inequality system is 
of two-block form  $A_1 x_1 + A_2 x_2 \le b$, 
and the variable indices for $x_1$, say $1, 4, 6, 7$,
are listed in the input file as

\begin{tabular}{ccl}
\\ \hline
\multicolumn{3}{l} {\bf begin}\\
  $m$  & $d+1$  & {\bf numbertype}\\
  $b$  & $-A$   &\\
\multicolumn{3}{l} {\bf end}\\
\multicolumn{3}{l} {preprojection} \\ 
\multicolumn{3}{l} { $4 \quad 1 \quad 4 \quad 6 \quad7$ } \\ \hline
\\
\end{tabular}
    
Then, cdd+ will output the inequality system,
$A_1 x_1  \le b$, together with the list $R$ of extreme
rays of the homogeneous cone  $\{z:  z  \ge 0  \mbox{ and }  z^T A_2 = 0 \}$.  
Consequently, the inequality system
$\{ \quad r^T A_1  x_1  \le  r^T b : \quad r \in  R \}$
represents the projection of the original polyhedron onto
$x_1$-space with possible redundancy.  The default file names
for the inequality system output and the extremal ray output are
*sub.ine and *.ext, respectively if the input file is named *.ine.  

There is a supplementary  C program, called domcheck, 
written by F. Margot, EPFL, which generates quickly a minimal
(i.e. nonredundant) system from these two outputs.
This program can be obtained from the standard ftp site for cdd.
\end{description}

\section{How to Use}  \label{HOWTO}

The program hardly has any user interface.  Once you have compiled
executable files,   {\em cddf+\/} and {\em cddr+\/} (see Section~\ref{CAUTIONS}), 
and once you create an input file,
say, {\em test.ine\/}, you have basically two ways to run the program.
The simplest way is just to run the program with 
\begin{verbatim}
  % cddf+ test.ine
\end{verbatim}
or, if you want to compute with rational (exact arithmetic)
\begin{verbatim}
  % cddr+ test.ine
\end{verbatim}
Then the program will open necessary output files with
default file names as shown in Table~\ref{FIG:files} 
and output the requested results.

\begin{table}[ht] 
\begin{center}
\begin{tabular}{|l||l|l|}   \hline
                       &  \multicolumn{2}{c|} {Input File Format }\\
options         &   \multicolumn{1}{c|}  {H-format (*.ine)} & \multicolumn{1}{c|}  {V-format (*.ext)} \\ \hline
conversion             &  *.ext   & *.ine  \\
incidence                &  *.ecd  &  *.icd \\
input\_incidence   &  *.iad  &  *.ead \\
adjacency   &  *.ead  &  *.iad \\
input\_adjacency   &  *.iad  &  *.ead \\
maximize/minimize   &  *.lps  &  non applicable \\
facet\_listing   &  *.fis  &  non applicable \\
vertex\_listing   &  non applicable  & *.vis \\
tope\_listing   &  *.tis  &  non applicable \\
verify\_input    & *.solved &   *.solved \\
preprojection  &  *sub.ine  &  non applicable \\
                          &  and *.ext  &                              \\ \hline
\end{tabular}
\end{center}
\caption{Default  extensions for output files} 
\label{FIG:files}
\end{table}


\bigskip  \noindent
If you want to specify the output file names
different from default, simply run the program by
\begin{verbatim}
  % cddf+   (or cddr+)
\end{verbatim}
and input desired file names at each of file name requests.
Even after you run cdd this way, one can 
change to the automatic mode by inputing the input file 
name with additional semicolon, e.g. ``test.ine;''.

To test cdd+, it is suggested to run cdd+ with sample input files
which are stored in subdirectories {\bf ine} and {\bf ext}.

\section{Source Files and Compilation}  \label{CAUTIONS}

\begin{itemize}
\item[(1)] [Files and Compilation] The source files for distribution are

\begin{tabular}{ll}
     cdd+.readme  & The readme file\\
     cdd.C       & C++ main source file\\
     cddarith.C  & C++ main arithmetic code\\
     cddpivot.C  & C++ pivot operation arithmetic code\\
     cddio.C     & C++ IO code\\ 
     cddrevs.C   & C++ reverse search code\\ 
     cdd.h       & The header file for cdd.C\\
     cdddef.h    & cdd+ definition file (whose two lines are to be edited by user)\\
     cddtype.h   & cdd+ arithmetic type definition file\\
     cddrevs.h   & The header file for cddrevs.C\\
     setoper.C   & C++ library for set operation\\
     setoper.h   & The header file for setoper.C \\
     cddman.tex  & Latex 2$\epsilon$ source file of cdd+ Reference Manual\\
     cddman.bbl  & bibliography file of cdd+ Reference Manual\\
     cddHISTORY  & brief description of changes made at each updates\\
     ine         & A subdirectory containing sample inequality input  files\\
     ext         & A subdirectory containing sample point/ray input files\\
     Makefile  & makefile for cddf+ and cddr+\\
     get\_essential  & gawk script for facet\_listing and vertex\_listing \\
     COPYING     & GNU GENERAL PUBLIC LICENSE\\
\end{tabular}

\noindent
For compilation of cdd+, one needs a recent (2.6.0 or higher) gcc compiler
and g++-library.   Once gcc and g++-library are installed, 
please edit Makefile according to the setup of a GNU gcc compiler and g++-library,
 and type
\begin{verbatim}
   % make all
\end{verbatim}
which creates two executables, cddr+ and cddf+.  The executable
cddr+ computes with rational (exact arithmetic) and cddf+ computes
with floating-point arithmetic.  If you want to create only one of
them, use ''make cddf+'' or ''make cddr+''.  Once these executables are 
created one might want to remove all object files *.o by
\begin{verbatim}
   % rm *.o
\end{verbatim}

We experienced some problems with older versions of gcc.    Also, be aware that
gcc and g++-library that come with NEXTSTEP 3.2 have bugs in the Rational library.
Please use gcc and g++lib on the newest version NEXTSTEP 3.3, or build a recent
gcc and g++library on older systems.
Generally, cdd+ seems to be most stable when compiled with gcc-2.6.3 
and libg++-2.6.2. 

Note that cddr+ reads Polyhedra data in integer or rational
number type, while cddf+ reads data in integer, rational and real number type.
When cddf+ reads integer or rational numbers, it first converts them
to floating point numbers and computes with floating-point arithmetic.

\item[(2)] [Recompilation] The first two constants in the program cdddef.h are to be 
changed by the user if necessary, and the program must be recompiled
each time after any change is made.   These constants are simply
to specify the largest size of acceptable input data $(b, -A)$: 
\begin{verbatim}
#define MMAX    5001  /* USER'S CHOICE: max row size of A plus one */
#define NMAX    101   /* USER'S CHOICE: max column size of A plus one */
\end{verbatim}
If this input data has $m$ rows and $d+1$ columns, then in the program,
MMAX should be at least $m+1$ and NMAX should be at least
$d+1$.  Although it has no sense to set the sizes MMAX and NMAX much larger
than necessary, the program only creates spaces for MMAX+NMAX pointers
and uses only necessary storage space for each input, and
thus large MMAX and NMAX won't be too harmful.

Unlike the pascal version pdd, one can set the size MMAX as large as one
wants.  It is no more restricted by the SET TYPE element sizes of 
usual Pascal compilers.

\item[(3)] [TURBO/THINK C Users]   Since cdd+ uses the GNU gcc libraries,
it cannot be compiled with other compilers.  Use  the ANSI C program cdd
instead which can be found in the same ftp site as cdd+.
\end{itemize}


\section{Some  Useful Tips for Usage}  \label{TIPS}

The computation is done by floating point arithmetic in cddf+ and
done by rational arithmetic in cddr+.    Since cddr+ runs much slower,
use it when you need to make sure that the output is correct.

Clearly, there is no guarantee 
that the program cddf+ outputs the correct result.  
However cddf+ seems to work correctly 
for many different types of polyhedra if one
carefully prepares input data files.   The followings
are some useful tips  for  input data preparation to 
avoid badly behaving computations with cddf+.

\begin{itemize}

\item  In cddf+, any real value is considered as zero if its absolute value is
less than $10^{-5}$.  Since the computation is performed with double precision
arithmetic, the correctness of zero recognition depends greatly on {\bf how
accurate\/} the input matrix $(b- A)$ is.  For example, you should never use
$0.9999$ for the value $1$.  Just use the correct value $1$ as it is.
Unlike many LP softwares, perturbation of data
can cause some serious problems.  If you want to perturb your data (e.g. right
hand side) for some reason, do it with large enough constants, say of order
$10^{-3}$.


\item If your matrix contains some irrational number, say
$\sqrt{5}$, please use an approximation which is correct in at least {\bf ten\/} digits,
i.e.  $2.236067977$.   See the sample input file  reg600-5.ine in the ine subdirectory.
 
\item  For the same arithmetic reason, please try to scale your input matrix
as even as possible by multiplying appropriate constants to some rows and
columns .   The program cdd does not perform any scaling before
computation.

\end{itemize}

\section{Bugs}  \label{BUGS}
\begin{itemize}

\item When the input system is a homogeneous system of linear inequalities
(i.e. the right hand side vector $b$ is a zero vector) and the homogeneous
cone determined by the system is pointed (i.e. the origin is a vertex) , 
the program cdd does not output this unique vertex.

\item Tope enumeration requires much storage space when the exact
computation is applied.  Currently we do not know how this happens
although it is certain that it is something to do with Rational class
library of g++. In future, this problem will be hopefully eliminated.

\end{itemize}

\section{FTP site}  \label{FTP}
An anonymous \htmladdnormallink{ftp}
{ftp://ifor13.ethz.ch/pub/fukuda/cdd/} site for the programs is set at:
\begin{verbatim}
   ftpsite:  ifor13.ethz.ch (129.132.154.13)
   directory: pub/fukuda/cdd
   filename: cdd+-***.tar.gz
\end{verbatim}
Since the file is compressed binary file, it is necessary to use binary mode for
file transfer.

\section{Other Userful Codes}  \label{CODES}
There are several other useful codes available for vertex enumeration and/or
convex hull computation  such as lrs, qhull, porta and irisa-polylib.
The pointers to these codes are available at
\begin{enumerate}
 \item  Avis'  lexicographic reverse search code lrs:\\
\htmladdnormallink{ftp://mutt.cs.mcgill.ca/pub/C/}
 {ftp://mutt.cs.mcgill.ca/pub/C/}.
 \item Geometry Center Software List by Nina Amenta:\\
\htmladdnormallink{http://www.geom.umn.edu/software/cglist/}
{http://www.geom.umn.edu/software/cglist/}\\
   (look for ''arbitrary dimensional convex hull''),
 \item Linear Programming FAQ by John Gregory:\\
\htmladdnormallink{http://www.skypoint.com/subscribers/ashbury/linear-programming-faq.html}
{http://www.skypoint.com/subscribers/ashbury/linear-programming-faq.html}\\
 (look for ''convex hull''),
 \item ZIB Berlin polyhedral software list:\\
 \htmladdnormallink{ftp://elib.zib-berlin.de/pub/mathprog/polyth/index.html}
{ftp://elib.zib-berlin.de/pub/mathprog/polyth/index.html}.
\end{enumerate}


\section*{Acknowledgements.} 
I am  grateful to Prof. Th. M. Liebling (EPFL, Switzerland)  who
provided me with an ideal opportunity to visit the department
for the academic year 1993-1994.  Without his 
generous help and encouragement,
the present form of this program would not have existed.
There are many people who helped me to improve cdd,  in particular,
I am indebted to Dr. Alain Prodon (EPFL) , Dr. Francois Margot
(UBC, Canada) , Dr. Henry Crapo (INRIA, France),
Dr. Alexander Bockmayr (Max Planck Institute, Germany), 
Mr. David Bremner (McGill University, Canada),
Dr. George Fankhauser (ETH Zurich) and
Dr.  Matthew Saltzman (Clemson University, U.S.A.). 

Finally, but not least, I would like to thank Prof. H.-J. L\"uthi
(ETH Zurich) who has made the current new development (cdd+)
possible.  Some important new features, including the exact computation
and the tope listing, have been added to cdd with his support.

\bibliographystyle{alpha}

\bibliography{fukuda1,fukuda2}

\end{document}


